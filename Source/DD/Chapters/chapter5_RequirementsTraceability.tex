\providecommand{\mainpath}{..} % Command to retrieve the path of the main file. It must be defined before documentclass.

\documentclass[\mainpath/main]{subfiles}
\begin{document}

\chapter{Requirements Traceability}
\label{RequirementsTraceability}
This chapter explains how the functional requirements from the Requirements Analysis And Specification Document (RASD) have been designed and described in this Design Document (DD).\\
Every section of this chapter refers to a functional requirement.

\section{Registration}
The section \ref{RASD:requirements:Registration} of the RASD describes the registration functionality. A visitor can register himself to myTaxyService only if he's new to the system, i.e. doesn't exist a user with the same tax code.\\
In the DD we have satisfied this functional requirement, we have described the sequence of actions in the Sequence Diagram of \autoref{ArchitecturalDesign:SD_Registration} and represented the user interface to register both via \gls{ma} and \gls{ws} in \autoref{UI:RegistrationAndLogin}.

\section{Login}
The section \ref{RASD:requirements:Login} of the RASD describes the login functionality. A visitor can login into myTaxyService only if he's registered.
In the DD we have satisfied this functional requirement, we have described the sequence of actions in the Sequence Diagram of \autoref{ArchitecturalDesign:SD_Login} and represented the user interface to login both via \gls{ma} and \gls{ws} in \autoref{UI:RegistrationAndLogin}.

\section{Personal Information Management}
The section \ref{RASD:requirements:PersonalInformationManagement} of the RASD describes the personal information management functionality. A logged user can modify some of his personal information: the e-mail, the password and the city of residence.\\
In the DD we have satisfied this functional requirement, we have described the sequence of actions in the Sequence Diagram of \autoref{ArchitecturalDesign:SD_ProfileManagement} and represented the user interface to manage personal information both via \gls{ma} and \gls{ws} in \autoref{UI:PersonalInformationManagement}.

\section{Ask for a Zerotime Ride}
The sections \ref{RASD:requirements:ZerotimeReservationMA} and \ref{RASD:requirements:ZerotimeReservationWS} of the RASD describe the functionality of asking for a zerotime ride both via \gls{ma} and via \gls{ws}. A logged user can ask for a zerotime ride.\\
In the DD we have satisfied this functional requirement, we have described the sequence of actions in the Sequence Diagram of \autoref{ArchitecturalDesign:SD_Zerotime} and represented the user interface to ask for a zerotime ride both via \gls{ma} and \gls{ws} in \autoref{UI:ZerotimeRide}.

\section{Book a Future Ride}
The section \ref{RASD:requirements:FutureReservation} of the RASD describes the functionality of booking a future ride both via \gls{ma} and via \gls{ws}. A logged user can book a future ride.\\
In the DD we have satisfied this functional requirement, we have described the sequence of actions in the Sequence Diagram of \autoref{ArchitecturalDesign:SD_FutureRide} and represented the user interface to book a future ride both via \gls{ma} and \gls{ws} in \autoref{UI:FutureRide}.

\section{Accept or Deny a Ride}
The section \ref{RASD:requirements:ZerotimeReservationMA} of the RASD also describes within the functionality of the Driver to accept or deny a ride, via \gls{ma}. A logged driver can accept or deny a request for a ride when he receives it.\\
In the DD we have satisfied this functional requirement and we have represented the user interface to accept or deny a ride via \gls{ma} in \autoref{UI:DriverFunctionalities}.

\section{Start Waiting Time}
The section \ref{RASD:requirements:StartWaitingTime} of the RASD describes the functionality of starting the waiting time via \gls{ma}. A logged driver notifies the system that he's available in an area and waiting for a ride and he is added to the queue of the area.\\
In the DD we have satisfied this functional requirement, we have described the sequence of actions in the Sequence Diagram of \autoref{ArchitecturalDesign:SD_StartWaitingTime} and represented the user interface to start waiting time via \gls{ma} in \autoref{UI:DriverFunctionalities}.

\section{Work Shifts Management}
The section \ref{RASD:requirements:WorkShiftsManagement} of the RASD describes the functionality of management of the work shifts via \gls{ma}. A logged driver can add or delete a work shift in the week days.\\
In the DD we have satisfied this functional requirement, we have described the sequence of actions in the Sequence Diagram of \autoref{ArchitecturalDesign:SD_WorkShiftsManagement} and represented the user interface to manage the work shifts via \gls{ma} in \autoref{UI:DriverFunctionalities}.

\section{Check the Reservations}
The section \ref{RASD:requirements:CheckReservations} of the RASD describes the functionality of checking the reservations (both zerotime and future rides) both via \gls{ma} and via \gls{ws}. A logged user can view the his reservations, modify the date or cancel them.\\
In the DD we have satisfied this functional requirement, we have described the sequence of actions in the Sequence Diagram of \autoref{ArchitecturalDesign:SD_CheckReservation} and represented the user interface to check the reservations both via \gls{ma} and \gls{ws} in \autoref{UI:OtherUserFunctionalities}.

\section{Read the Alerts}
The section \ref{RASD:requirements:ReadAlerts} of the RASD describes the functionality of reading the received alerts both via \gls{ma} and via \gls{ws}. A logged user can only view his alerts, he cannot edit them.\\
In the DD we have satisfied this functional requirement and we have represented the user interface to read the alerts both via \gls{ma} and \gls{ws} in \autoref{UI:OtherUserFunctionalities}.



\end{document}