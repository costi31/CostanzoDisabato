% !TEX root = ../dd.tex

\documentclass[../dd]{subfiles}
\begin{document}

\chapter{Architectural Design}
\label{architectural_design}

% Command to be executed after the starting of every chapter
\setmyfancystyle
% ----------------

In this chapter the complete architecture of myTaxiService is shown with various levels of description. In the \autoref{ArchitecturalDesign:high_level} there is a global view and the interactions between all the components are described.\\ 
The data tier is illustrated in \autoref{ArchitecturalDesign:component} with all related policies and entities. Then the other tiers are characterized using special diagrams.\\
In the \autoref{ArchitecturalDesign:deploy} I don't know yet what we have to write.\\
in \autoref{ArchitecturalDesign:runtime} the view level is defined. The interactions between all kinds of user and the system are described using UX diagrams and sequence diagrams that display the order in which each screen is visualized. Besides, the mockups of these screens are shown in %\autoref{UI}.
\\
% component interfaces?
Finally, in the \autoref{ArchitecturalDesign:design_patterns} the design patterns used to develop myTaxiService are described first in general case. After that, all the changes needed to adapt this design patterns to our system are characterized.


\end{document}


\section{High level components and their interaction}
\label{ArchitecturalDesign:high_level}
afsjfasjkf

\section{Component view}
\label{ArchitecturalDesign:component}
asdkjfjasfk

\section{Deployment view}
\label{ArchitecturalDesign:deploy}
adshjkfasjkf

\section{Runtime view}
\label{ArchitecturalDesign:runtime}
jkfagfsjakgkjasghakjghjsdfglhljas

\section{Selected architectural styles and patterns}
\label{ArchitecturalDesign:design_patterns}
EVENTBASED per ora