% !TEX root = rasd.tex
\usepackage[printonlyused]{acronym}
\usepackage{amsfonts}
\usepackage{amsmath}
\usepackage{amssymb}
\usepackage[english]{babel}
\selectlanguage{english}
\usepackage[footnotesize]{caption}
\usepackage[T1]{fontenc}
\usepackage{float}
\usepackage{graphicx}
\usepackage{indentfirst}
\usepackage[latin1]{inputenc}
\usepackage{longtable}
\usepackage{pdfpages}
\usepackage{placeins}
\usepackage{textcomp}
\usepackage{fancyvrb}
\usepackage{fancyhdr}
\usepackage{lastpage}
\usepackage{enumitem}
\setlist{nosep}

% Comando creato da me per impostare rapidamente lo stile della pagina
\newcommand{\setmyfancystyle}{
	\thispagestyle{fancy}
	\fancyhead{}
	\fancyfoot{}
	\renewcommand{\headrulewidth}{0pt} % elimina la linea orizzontale in cima alla pagina
	\lhead{} \chead{} \rhead{}
	\lfoot{} \cfoot{}
	\rfoot{\thepage}
}

%\usepackage{alloy}

\usepackage[pdftex]{hyperref}
\hypersetup{
breaklinks=true,
pdfborder={0 0 0},
pdfstartview={FitH},
pdfpagemode={UseOutlines},
pdftitle={},
pdfauthor={},
pdfsubject={},
pdfkeywords={}
}

\usepackage{listings}
\lstset{
tabsize=8,
numbers=left,
breaklines,
breakatwhitespace,
basicstyle=\scriptsize\ttfamily,
keywordstyle=\bfseries,
emphstyle=\ttfamily\itshape\underbar,
showstringspaces=false,
frame=single,
captionpos=b,
aboveskip=1em
}







\renewcommand*{\lstlistingname}{Listed}
\renewcommand*{\lstlistlistingname}{List of Listed}
\renewcommand{\figurename}{Figure}


\renewcommand{\rmdefault}{ppl}
\usepackage[scaled]{helvet}
\usepackage{courier}
\normalfont

\frenchspacing
\linespread{1.25}


\newcommand{\eg}{\textit{e.g.,}~}
\newcommand{\ie}{\textit{i.e.,}~}

