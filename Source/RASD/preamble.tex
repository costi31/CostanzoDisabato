% !TEX root = rasd.tex
\usepackage[printonlyused]{acronym}
\usepackage{amsfonts}
\usepackage{amsmath}
\usepackage{amssymb}
\usepackage[english]{babel}
\selectlanguage{english}
\usepackage[footnotesize]{caption}
\usepackage[T1]{fontenc}
\usepackage{float}
\usepackage{graphicx}
\usepackage{indentfirst}
\usepackage[latin1]{inputenc}
\usepackage{longtable}
\usepackage{pdfpages}
\usepackage{placeins}
\usepackage{textcomp}
\usepackage{fancyvrb}
\usepackage{fancyhdr}
\usepackage{lastpage}
\usepackage{enumitem}
\setlist{nosep}

% Comando creato da me per impostare rapidamente lo stile della pagina
\newcommand{\setmyfancystyle}{
	\thispagestyle{fancy}
	\fancyhead{}
	\fancyfoot{}
	\renewcommand{\headrulewidth}{0pt} % elimina la linea orizzontale in cima alla pagina
	\lhead{} \chead{} \rhead{}
	\lfoot{} \cfoot{}
	\rfoot{\thepage}
}
%---

\usepackage[pdftex]{hyperref}
\hypersetup{
	breaklinks=true,
	pdfborder={0 0 0},
	pdfstartview={FitH},
	pdfpagemode={UseOutlines},
	pdftitle={},
	pdfauthor={},
	pdfsubject={},
	pdfkeywords={}
}

\usepackage{listings}
\lstset{
	tabsize=8,
	numbers=left,
	breaklines,
	breakatwhitespace,
	basicstyle=\scriptsize\ttfamily,
	keywordstyle=\bfseries,
	emphstyle=\ttfamily\itshape\underbar,
	showstringspaces=false,
	frame=single,
	captionpos=b,
	aboveskip=1em
}



\renewcommand*{\lstlistingname}{Listed}
\renewcommand*{\lstlistlistingname}{List of Listed}
\renewcommand{\figurename}{Figure}


\renewcommand{\rmdefault}{ppl}
\usepackage[scaled]{helvet}
\usepackage{courier}
\normalfont

\frenchspacing
\linespread{1.25}

% Alloy
\usepackage{xcolor}
\usepackage{alloy-style}
%---


\newcommand{\eg}{\textit{e.g.,}~}
\newcommand{\ie}{\textit{i.e.,}~}



% Glossary package, with the options to show it as a numbered section and to show it in the table of contents

\usepackage[nonumberlist, section, numberedsection, toc, style=altlist]{glossaries}

\newglossaryentry{ws}
{
	name=WS,
	description={acronym for the website of the system}
}
\newglossaryentry{ma}
{
	name=MA,
	description={acronym for the mobile application}
}
\newglossaryentry{visitor}
{
	name=Visitor,
	text=visitor,
	plural=visitors,
	description={a person who accesses to the \gls{ws} or opens the \gls{ma}, but it isn't enrolled or logged in the system (usually it is his first visit)}
}
\newglossaryentry{user}
{
	name=User,
	text=user,
	plural=users,
	description={a person who is logged in the system}
}
\newglossaryentry{passenger}
{
	name=Passenger,
	text=passenger,
	plural=passengers,
	description={a person who is using the service in the current moment}
}
\newglossaryentry{gps}
{
	name=GPS,
	description={a system able to calculate the correct position of each driver}
}
\newglossaryentry{map}
{
	name=Map,
	text=map,
	plural=maps,
	description={an informatics representation of the taxi company's city, that can be used either by the user and by the system (also with \gls{gps})}
} 
\newglossaryentry{venue}
{
	name=Venue,
	text=venue,
	plural=venues,
	description={(departure or destination venue): a valid address that identifies a precise point on the map (and in the real city)}
}
\newglossaryentry{zerotime}
{
	name=Zerotime,
	text=zerotime,
	description={a neologism (due to precision it is a composed word) that indicates an action that the system tries to provide immediately}
}
\newglossaryentry{future}
{
	name=Future,
	text=future,
	description={a temporal period that in our project is between 1 hour and 30 days from the current day}
}
\newglossaryentry{delay-time}
{
	name=Delay-time,
	text=delay-time,
	description={the maximum time of waiting ensured by the system to each user. It is calculated from the ride's established time}
}
\newglossaryentry{driver}
{
	name=Driver,
	text=driver,
	plural=drivers,
	description={an employer of the taxi's company who drives the reserved taxi}
}
\newglossaryentry{ride}
{
	name=Ride,
	text=ride,
	plural=rides,
	description={word referred to any single service given by the taxi's company. The service is a single-passenger travel with a driver between two precise venues}
}
\newglossaryentry{reservation}
{
	name=Reservation or booking,
	text=reservation,
	plural=reservations,
	description={the action that allows an user to take a \gls{ride} in \gls{zerotime} or in the \gls{future}}
}
\newglossaryentry{alert}
{
	name=Alert,
	text=alert,
	plural=alerts,
	description={a system internal message to inform about, ask (a \gls{driver}) and organize a single \gls{ride}}
}
	

\makeglossaries
