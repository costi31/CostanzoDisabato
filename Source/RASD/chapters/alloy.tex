% !TEX root = ../rasd.tex
\chapter{Alloy Modelling}
\label{alloy}

In this chapter the consistency of the proposed Class Diagram will be tested via Alloy Analyzer. The report that follow is composed by the code used to describe the model and an example of world generated by our predicates.

\section{Alloy Code}
Here the code used is presented.

\lstdefinelanguage{Alloy}
{morekeywords={abstract, all, and, as, assert, but, check, disj, else,
  exactly, extends, fact, for, fun, iden, if, iff, implies, in, Int,
  int, let, lone, module, no, none, not, one, open, or, part, pred,
  run, seq, set, sig, some, sum, then, univ},
sensitive=true,
morecomment=[l]{--},
morecomment=[l]{//},
%morecomment=[s]{/*}{*/},
morecomment=[l]{/***},
tabsize=2,
columns=fullflexible,
}

\lstdefinestyle{customc}{
  belowcaptionskip=1\baselineskip,
  breaklines=true,
  frame=L,
  xleftmargin=\parindent,
  language=C,
  showstringspaces=false,
  basicstyle=\footnotesize\ttfamily,
  keywordstyle=\bfseries\color{blue},
  commentstyle=\itshape\color{green!40!black},
  identifierstyle=\color{black},
  stringstyle=\color{orange},
}

\lstset{escapechar=@,style = customc}



\lstinputlisting[language=Alloy]{./code/alloyModelv2.als}


\section{Alloy Response}
Here the alloy response to the model is shown.
\begin{Verbatim}[commandchars=\\\{\},codes={\catcode`$=3\catcode`_=8}]
5 commands were executed. The results are:
   #1: No counterexample found. noSelfInvitation may be valid.
   #2: No counterexample found. noSneakInEvent may be valid.
   #3: No counterexample found. noGhostEvent may be valid.
   #4: \textbf{Instance found}. showEvent is consistent.
   #5: \textbf{Instance found}. show is consistent.
\end{Verbatim}

\subsection{Alloy World}
Here some examples generated by the description of the model is displayed.
\subsubsection{World 1}
\label{w1}
\begin{figure}
	\centerline{\includegraphics[scale=0.5, angle=90]{./Figures/alloyWorld1.png}}
	\caption{World created from the predicate showInvite}
	\label{aw1}
\end{figure}

In \figurename~\ref{aw1} is shown the world generated by Alloy Analizer via the execution of the predicate "showInvite". In this predicate the number of users and events is respectively constrained to be greater than 1 and 2. in this way the Analizer generates a significant populated world. It could be assume that the world generated under the imposed constraints is consistent.

\subsubsection{Word 2}
\begin{figure}
	\centerline{\includegraphics[scale=0.9, angle=90]{./Figures/alloyWorld2.png}}
	\caption{World created from the predicate show}
	\label{aw2}
\end{figure}

In \figurename~\ref{aw2} is shown the world generated by Alloy Analizer via the execution of the predicate "show". In this predicate there are no constraints. The predicate generates a world that respects the given constraints where each entities has at maximum 3 instances.

\subsubsection{World 1 bis}
\begin{figure}
	\centerline{\includegraphics[scale=0.5, angle=90]{./Figures/alloyWorld3.png}}
	\caption{World created from the predicate showInvite with skolemization option enabled}
	\label{aw3}
\end{figure}

in \figurename~\ref{aw3} is shown the same world presented in \ref{w1}, but in this case the skolemization option is enabled.

\section{Alloy Result Evaluation}
Observing the generated world in \figurename~\ref{aw3}, the existence of some skolemization functions, that are disabled in the model presented in \figurename~\ref{aw1}, can be extrapolated.
One first conclusion could be simply done from the Alloy code presented. It's likely that some "fact" over the calendar behaviour has inside unnecessary existential quantifiers over the variable "c" (that refers to Calendar). But the need to introduce this quantifiers is directly consequence of the model representation that is been chosen to adopt.
So it's safe to say that the skolem function are also consequences of the cycles introduced in paragraph concerning the UML Class Diagram. This proves the hypothesis made in the chapter about the data redundancy, showing that the same relation can be obtained without the use of the class Calendar. In spite of that is still useful to regroup connected data and to a simpler visual comprehension,  at least in this first part of the project. During the following project phases the elimination of these redundancies could be taken into account. 

\acresetall