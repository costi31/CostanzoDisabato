\providecommand{\mainpath}{..} % Command to retrieve the path of the main file. It must be defined before documentclass.

\documentclass[\mainpath/main]{subfiles}
\begin{document}

\chapter{Assignment} % First chapter
\label{Assignment}

% Command to be executed after the starting of every chapter
\setmyfancystyle
% ----------------

In this chapter we will present the class that has been assigned to us. First, the lines of code to be analysed are presented without any comment. Afterwards, a brief description of the class role and function is presented. Obviously this is made by us and it is based only on the code and on the documentation provided by the authors. 

\section{Assigned class}
\label{Assignment:AssignedClass}
The class assigned to us is called \textquotedblleft \textbf{SecurityMechanismSelector}\textquotedblright and is located in the following path relative to the root of Glassfish project:\\
\textit{
appserver/security/ejb.security/src/main/java/com/sun/enterprise/iiop/security/
}\\ \\
The package which the class belongs to is: \textit{com.sun.enterprise.iiop.security}\\
To present more clearly the class we include the first lines of code containing the documentation of the class and its declaration:

\includecodecap{108}{122}{Class documentation and declaration}

\subsection{Functional role}
As the javadoc states, this class is responsible for selecting the security mechanism based on target configuration and client policies.

TO BE CONTINUED...

\section{Assigned methods}
\label{Assignment:AssignedMethods}
The methods assigned to us are 4, that are presented below, with the entire code and with our description of their functional role.

\subsection{First method}
\label{Assignment:AssignedMethods:FirstMethod}
\includecodecap{963}{986}{First assigned method}
This method is supposed to return the subject of the current security context. A Subject represents a grouping of related information for a single entity, such as a person. Such information includes the Subject's identities as well as its security-related attributes (passwords and cryptographic keys, for example).

\end{document}