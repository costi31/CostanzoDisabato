\usepackage[pdftex]{hyperref} % Package for references
\hypersetup{
	breaklinks=true,
	pdfborder={0 0 0},
	pdfstartview={FitH},
	pdfpagemode={UseOutlines},
	pdftitle={},
	pdfauthor={},
	pdfsubject={},
	pdfkeywords={}
}

\usepackage{nameref}

\usepackage[footnotesize]{caption} % Adds the possibility to write text with the footnote size and sets a smaller size for the caption of images

\usepackage{geometry} % Required to change the page size to A4
\geometry{a4paper} % Set the page size to be A4 as opposed to the default US Letter

\usepackage{graphicx} % Required for including pictures

\usepackage[english]{babel}
\selectlanguage{english} % Sets to English the language of document

\usepackage[printonlyused]{acronym} %Deletes some blank pages

\usepackage[T1]{fontenc}	% Defines characters' type
\usepackage[latin1]{inputenc} 	% Defines admissible fonts

\usepackage{indentfirst} % Indents the first row of a paragraph

\usepackage{float} % Allows putting an [H] in \begin{figure} to specify the exact location of the figure
\usepackage{wrapfig} % Allows in-line images such as the example fish picture

\usepackage{lipsum} % Used for inserting dummy 'Lorem ipsum' text into the template

\usepackage{fancyvrb}

\usepackage{fancyhdr} % Used to customize the header and footer of pages, like page numbering

\usepackage{lastpage} % Used to obtain the number of pages

\usepackage{enumitem}
\setlist{nosep} % Deletes the space before an items list and after

% Custom command to reset the page header and footer style
% This command must be called every time after a begin document is called or after a new chapter has begun
\newcommand{\setmyfancystyle}{
	\pagestyle{fancy}
	\thispagestyle{fancy}
	\fancyhead{}
	\fancyfoot{}
	\renewcommand{\headrulewidth}{0pt} % deletes the horizontal line in the page header
	\lhead{} \chead{} \rhead{}
	\lfoot{} \cfoot{}
	\rfoot{\thepage}
}
%-----------

\renewcommand{\rmdefault}{ppl}
\usepackage[scaled]{helvet}
\usepackage{courier}
\normalfont

\frenchspacing
\linespread{1.25} % Line spacing

%\setlength\parindent{0pt} % Uncomment to remove all indentation from paragraphs

\graphicspath{{Figures/}{../Figures/}} % Specifies the directory where pictures are stored

\usepackage{subfiles} % Allow to manage the document as different files (useful for working in two at the same time

\usepackage{xr} % Allows to use references to external latex documents