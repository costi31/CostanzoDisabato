\providecommand{\mainpath}{..} % Command to retrieve the path of the main file. It must be defined before documentclass.

\documentclass[\mainpath/main]{subfiles}
\begin{document}

\chapter{Tools and Test Equipment Required} % First chapter
\label{ToolsAndTestEquipmentRequired}

% Command to be executed after the starting of every chapter
\setmyfancystyle
% ----------------

In this chapter we are going to present the tools that will be used to perform the integration test. They will be presented with their name, a link to their reference documents in the \autoref{Introduction} and a short explanation concerning the reasons why and the situations where they will be used.\\
The first one is \textit{Mockito}, available at the website (it is a GitHub repository) \url{http://mockito.github.io}. This tool is especially useful to design and to implement the program stubs, described in the \autoref{ProgramStubsAndTestDataRequired}, because it allows to specify predefined \textit{answer} to each method calling with a few lines of code.

\begin{figure}[!ht]
	\centering
	\includegraphics[height = 2cm]{mockito}
\end{figure}

The second tool is \textit{Arquillian}, developed to make easy the integration tests. Hence, its use will allow us to perform the integration tests.\\
Further information about this tool are available at its website, \url{http://arquillian.org}.

\begin{figure}[!ht]
	\centering
	\includegraphics[height = 2cm]{arquillian}
\end{figure}

Finally, an important test about the functionalities of \textit{myTaxiService} is about its capacities and response times during normal or intense executions. This kind of test can be performed by using \textit{Apache JMeter}\footnote{Further information about this tool are available at \url{http://jmeter.apache.org}.} tool and with a little different client stubs w.r.t. the one defined in the \autoref{ProgramStubsAndTestDataRequired} (able to return random, but valid data instead of returning always the same ones). Despite the importance of this kind of tests, they are not argument of this document.

\begin{figure}[!ht]
	\centering
	\includegraphics[height = 2cm]{jmeter}
\end{figure}




\end{document}